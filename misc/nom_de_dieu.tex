\documentclass[12pt,a4paper]{article}
\usepackage[utf8]{inputenc}
\usepackage[T1]{fontenc}
\usepackage[francais]{babel}
\usepackage{graphicx}
\usepackage{url}


\author{Pierre Ghazarian, 6A}
\title{Nom(s) de dieu}

%\oddsidemargin=-1cm
%\setlength{\textwidth}{6.5in}
%\addtolength{\voffset}{-5pt}
\linespread{1.3}

\begin{document}

\maketitle

%\part*{Introduction}

\part{Le titre}

Tout comme certains des \og{}Pères Fondateurs\fg{} des États-Unis, dont Benjamin Franklin et Thomas Jefferson,  je me déclare, par simplicité, déiste. Mais contrairement à Benjamin Franklin qui déclarait que {\og{}I believe there is one Supreme most perfect being. (...) I believe He is pleased and delights in the happiness of those He has created; and since without virtue man can have no happiness in this world, I firmly believe He delights to see me virtuous.\fg{}\footnote{Je crois qu'il y a un Être suprême le plus parfait. (...) Je crois qu'Il est satisfait et se réjouit dans le bonheur de ceux qu'Il a créé ; et puisque sans la vertu l'homme ne peut atteindre le bonheur en ce monde, je crois fermement qu'Il se réjouit de me voir vertueux (traduction libre).}  dans ses \emph{Articles of Belief and Acts of Religion}, en 1728, je ne prétends pas à un tel degré de certitude.\\ 

Je ne prétends pas non plus à la compréhension des phénomènes naturels, et en particulier de la vie, et, en cela, je crois à une création divine possible. Je suis donc déiste agnostique. Je ne suis pas certain de l'existence de ce dieu, ou pour l'appeler autrement, de cette entité suprême. À l'instar de Voltaire, par ailleurs, je tiens à combattre l'intolérance sous toutes ses formes, et en conséquence, tout ce qui pourrait nuire à la liberté de qui que ce soit, à n'importe quel degré.\\

De plus, je ne sais pas si les phénomènes que je ne comprends pas et que j'entoure d'une aura mystique ne seront pas un jour expliqués rationnellement, d'où mon agnosticisme.\\

Ceci explique mon choix de titre : \og{}dieu\fg{}, dont je ne suis pas convaincu de l'existence, n'est pas le Dieu des chrétiens, mais une force que je n'arrive pas à cerner complètement. En cela, \og{}il\fg{} ne mérite pas de capitalisation de la première lettre de son nom, et il est seul, mais pourrait porter plusieurs noms. Ainsi, les Jedis de l'univers fictif de \emph{Star Wars} l'appellent-ils \og{}la Force\fg{}, dont la définition est la suivante : \emph{la Force est une sorte de champ d'énergie généré par tout être vivant et qui englobe tout l'univers. Ce principe peut se rapprocher de certaines religions réelles comme le Shinto japonais, ou de concepts druidiques celtes, et a de fortes similarités avec le Taoïsme. En d'autres termes, la Force est considérée comme provenant d'un amalgame de religions et philosophies, se déclinant comme une métaphore de la spiritualité}\footnote{\url{http://fr.wikipedia.org/wiki/Jedi#La_Force}, consulté le premier mai 2012 à 16h53}.\\

En cela, ma vision de Dieu a évolué, car, au départ, je reniais farouchement la possibilité qu'une entité quelconque ait pu avoir quoi que ce soit comme contrôle sur la matière, reléguant cette habilité au seul hasard, hasard que j'ai par la suite déterminé être l'\oe{}uvre de facteurs incompréhensibles et innombrables, ceci s'approchant de la théorie du chaos, par la suite, j'ai compris que tout pouvait être expliqué, et qu'il valait mieux se méprendre plutôt que d'essayer d'admettre la possibilité que l'entropie l'emporte sur le reste, régissant ainsi tout événement, personne ne pouvant déterminer la relation de cause à effet exacte. \\

Mes conclusions ont par la suite été que la vie ne pouvait elle-même être issue du hasard, tout simplement, l'entropie ne pouvant générer que du désordre supplémentaire et la vie représentant un ensemble de phénomènes ordonnés, elle ne pouvait être le fait de génération de hasard, donc, je le suppose, non spontanée. Je suppose que des connaissances plus pointues en sciences pourraient me prouver le contraire, d'ailleurs.\\

J'ai plus tard appris de nouvelles notions de biologie, et me suis rendu compte de la complexité de la vie, et surtout de l'intelligence résidant derrière l'évolution. Car même si l'évolution se fait en partie grâce au hasard, je ne suis pas tout à fait d'accord avec cette vision d'essai et d'erreur en tant qu'unique solution au problème, l'adaptabilité des êtres vivants et leur résistance à toute contrariété n'étant plus à démontrer, car, en effet, la vie résiste très bien aux catastrophes nucléaires, comme le démontre l'état de la faune et de la flore dans la région avoisinant la centrale de Tchernobyl\footnote{\url{http://www.futura-sciences.com/fr/doc/t/physique/d/tchernobyl-consequences-accident-nucleaire_251/c3/221/p6/}, consulté le premier mai 2012 à 17h14)}, qui, même si elle a été éprouvée, a été capable de survivre à un environnement hautement toxique, le fait qu'elle se soit adaptée si rapidement est pour moi un signe plausible d'une certaine forme d'intelligence résidant dans le phénomène de la vie lui-même, pas de quoi me déclarer stoïque, donc.\\

\part{L'image}

\includegraphics[width=\textwidth]{ect.jpg}

Photographie d'un patient recevant un traitement par électrochocs, Worchester, Massachussets, août 1949, par Herbert Gehr, l'utilisation de cette image n'est autorisée que pour un usage non commercial par \textsc{LIFE}.\\

Si j'ai choisi cette image, c'est d'abord par provocation, en effet, j'espère que la vision de cette image suscite chez qui que ce soit la voyant un immédiat sentiment d'horreur : sur un lit se tord un homme à qui on inflige des douleurs insupportables, ceci afin de le soigner de maladies qui sont pourtant très mal comprises, et dont les remèdes, et surtout la thérapie par électrochocs, sont constamment sujet à controverse \footnote{ \url{http://www.nmha.org/go/information/get-info/treatment/electroconvulsive-therapy-ect}}. Par extension, cette image représente très bien les abus potentiels de pouvoir que l'homme commet volontiers, ignorant la raison pour pouvoir torturer plus faible que lui, mais aussi, surtout, pour s'en prendre à ce qu'il ne comprend pas, qui l'effraie.\\

Pour moi, les maladies mentales sont un excellent \og{}excipient\fg{}, parmi d'autres, de l'aspect le plus sombre de la nature humaine : la soif de pouvoir, qui va jusqu'à la conviction profonde, chez certains individus, que leurs actes sont bons alors qu'ils nuisent de façon évidente à leur prochain. Cette irrationalité typiquement humaine ne me révolte pas, elle ne fait que me décourager d'un avenir où la raison puisse avoir une place prédominante. Même si le personnage m'afflige de son ridicule, je suis forcé d'avouer que je suis d'accord avec le Marquis de Sade qui dit un jour : \og{}Le pouvoir est par nature, criminel\fg{}. Certains autres personnages, comme Jacques Chirac et Henry Kissinger, hommes que je dénigre pour le fait même qu'ils aient pu penser pouvoir gouverner d'autres êtres humains, dirent respectivement : \og{}Le pouvoir ne se partage pas.\footnote{Cité dans \emph{Le Mariage blanc, Mitterrand-Chirac}, J.-M. COLOMBANI \& J.-Y. LHOMEAU, Grasset, octobre 1986}\fg{} et \og{}Power is the ultimate aphrodisiac.\footnote{\emph{New York Times}, 28 octobre 1973}\fg{}.  \\

Je suis bien conscient de l'impossibilité de l'absence d'abus de pouvoir autant que je suis conscient de l'impossibilité de l'absence du pouvoir tout court, rien n'empêche cependant d'envisager un modèle où l'abus est rendu impossible par un système totalement transparent et s'auto-corrigeant aussi vite qu'il commet une erreur, mais je suis d'autant plus conscient que de telles caractéristiques sont loin d'être humaines. Or, notre société, du moins, de mon point de vue, s'enlise dans la soumission à un schéma de vie banal et imperturbable, où toute contrariété est ignorée et laissée au pouvoir, en l'occurrence l'État. Cette méthode a pour inconvénient indéniable de laisser place à la corruption, l'humain étant entièrement, ou presque, corruptible, et ce, à tous les points de vue. Quelle certitude peut-on avoir qu'une personne qui nous est complètement étrangère n'agisse pas selon le seul appât du gain ? Si accorder sa confiance à un inconnu pour des aspects peu importants d'une vie est une mauvaise idée, accepter de céder ses droits fondamentaux à quelqu'un d'autre est de l'ordre de l'inconscience. Benjamin Franklin a d'ailleurs déclaré que \og{}Ceux qui sont prêts à abandonner une liberté fondamentale pour obtenir temporairement un peu de sécurité, ne méritent ni la liberté ni la sécurité.\footnote{\emph{An Account of Negotiations in London for Effecting a Reconciliation between Great Britain and the American Colonies} (1775), dans \emph{The Complete Works of Benjamin Franklin}, Benjamin Franklin, éd. G. P. Putnam’s Sons, 1887, t. 5, p. 513}\fg{}, en effet, pour moi, le pouvoir politique est une liberté fondamentale, et espérer pouvoir être représenté est l'abandonner pour un peu de sécurité intellectuelle : celle de ne pas devoir s'appliquer dans les affaires de son pays, ou autre entité administrative quelle qu'elle soit. \\

Et voilà pourquoi cette image représente si bien la société actuelle : en tant qu'humains, nous préférons le paisible confort de la lâcheté, et comme le dit si bien Descartes \og{}Il y a beaucoup plus de sûreté et plus d'honneur en la résistance qu'en la fuite\footnote{\emph{Les Passions de l'âme} (1649), dans \emph{Œuvres}, René Descartes, éd. F.-G. Levrault, 1824-1826, vol. 4, partie 3 (« Des passions particulières »), article CCXI, p. 211-212}.\fg{}, en effet, l'homme sur cette photographie a été abandonné par ses pairs à des ignorants sans scrupules préférant lui infliger de la douleur plutôt que de réfléchir à une méthode moins traumatisante de soins. Par extension, on peut considérer que nous préférons tout ce qui ne nous demande pas de réfléchir, de penser de façon critique, et surtout, de remettre en question ce qui nous permet d'atteindre le sacro-saint confort. Bien sûr, il y a des exceptions, mais une société n'est pas caractérisée par ses éléments les plus brillants, hélas.\\

\part{La phrase}

\og{}En comptant tous les dieux, demi-dieux, quarts de dieux... il y a déjà eu 62 millions de dieux depuis les débuts de l'humanité ! Alors, les mecs qui pensent que le leur est le seul bon...\fg{} -- Coluche\footnote{dans \emph{Pensées et anecdotes}, aux éditions le Cherche Midi, collection \og{}Les pensées\fg{} (1 octobre 1995)}\\

Cette phrase explicite pour moi très bien le besoin de ceux qui croient en un dieu, ou en Dieu, ou en dieux, de se justifier en tant que croyant, et, par la même occasion, de dénigrer quelque peu le(s) dieu(x) des autres. Il y a une sorte de compétition entre les religions, qui passe par le prosélytisme, la course à l'influence, et toutes sortes de luttes aussi diverses que variées, qui vont jusqu'à l'intolérance et, parfois même, la guerre. Ce besoin de justification n'est pas que religieux, mais la religion a le tort d'être partagée par beaucoup de monde, sous le même \og{}étendard\fg{}. C'est ainsi que tous les belligérants d'une guerre ressentent le besoin de se convaincre que Dieu est avec eux, et justifient leurs actions de la sorte, entre autres raisons. Et c'est ainsi que la religion peut servir au mal, même si son message initial était contraire.\\

Mais, d'une façon plus générale, le fait de croire qu'un dieu soit le bon est plutôt ridicule du fait de la ressemblance de tous les cultes religieux : tous adressent les mêmes questions métaphysiques, seules les réponses changent quelque peu.\\

Le fait que la religion soit si facilement rassurante, et donc lâche à mes yeux, ne me pousse pas au mépris des religieux, en effet, la tâche serait bien trop compliquée, et même si je n'ose pas me clamer tolérant, je ne tiens pas à raisonner qui que ce soit en la matière non plus. En réalité, je vois la religion comme un produit de mauvaise qualité, à tarif bien moins élevé que les autres, plutôt que comme un danger.\\

La raison en est simple : la religion tient à imposer des concepts en créant des dogmes : pour moi, ce système est contraire à la raison et promeut une forme d'oppression intellectuelle. Mais comme je l'ai mentionné plus tôt, je considère que l'homme préfère cet état de fait à la nécessité d'une réflexion profonde, et c'est pourquoi la religion a tant de succès : elle impose des théories là où il ne peut encore y en avoir de raisonnées.\\

Ainsi, je considère la religion comme une forme arriérée de philosophie, et comme une convention dépassée et inutile, mais, comme je l'ai dit plus tôt, je trouve encore plus inutile que la religion elle-même toute tentative de reconversion des fidèles, je crois en effet à la liberté individuelle avant de croire en quoi que ce soit d'autre par rapport au monde. Mais je me dois d'apporter une nuance à mes propos pour ne pas passer pour un cynique de bas étage, car si la religion est souvent arriérée, elle peut faire preuve de raison et enseigner des préceptes raisonnables, c'est parfois leur interprétation qui leur fait défaut.\\

Cette phrase ne représente pas que la religion mais aussi la tendance à vouloir affirmer ses croyances comme étant les meilleures de toutes (ce que je suis d'ailleurs en train de faire), une pratique assez courante qui, à une certaine échelle, peut devenir du conformisme, et ainsi, le groupe exercera une pression sur les nouveaux venus, qui devront se plier à cette pratique. Si je suis convaincu de quelque chose, et malgré ma conviction parallèle que je ne puis éduquer personne, je tiens à ce quelques autres humains soient d'accord avec moi, et en cela, je dois déroger à ma nature.\\

Par ailleurs, cette tendance à vouloir imposer des croyances est ce qui m'insupporte le plus dans notre société, et ce pourquoi je me protège autant que possible de toute forme de publicité, et tout autre discours manipulateur, et donc, hélas, de tout discours religieux, ou presque : j'ai toujours eu l'impression qu'on tentait de me vendre une solution plutôt que m'enseigner quoi que ce soit, la faute à des mauvais prêcheurs, peut-être.\\

Cette phrase aurait pu adresser le problème de façon plus large, mais la religion est un très bon exemple, et l'argument invoqué de façon elliptique est assez bien placé pour laisser au lecteur le choix de ce qu'il veut y mettre, je ne pense donc pas qu'on puisse sensiblement l'améliorer.\\

\part{Le symbole}

\includegraphics[width=\textwidth]{Ethernet_RJ45_connector_p1160054.jpg}

J'ai choisi comme symbole une prise mâle RJ-45, souvent utilisée pour relier des câbles eux-mêmes utilisés pour transporter le protocole Ethernet, qui constitue une bonne partie des deux premières couches d'Internet, pour représenter la liberté d'expression que ce réseau apporte depuis sa création à toute personne y ayant accès.\\

En effet, Internet est pour moi la plus grande invention de l'humanité : pour la première fois, tout le monde peut s'exprimer librement sans restrictions et sans moyens faramineux de diffusion, utilisant les possibilités techniques offertes par l'interconnexion d'ordinateurs, tout le monde peut partager ce qu'il peut numériser, tout cela grâce à des technologies abordables et aux spécifications ouvertes, donc reproductibles facilement et sans restrictions, et comme Vinton Cerf, un des pères de l'Internet Protocol,  l'a très bien résumé : \og{}The Internet lives where anyone can access it\footnote{Internet vit là où quelqu'un peut y accéder (traduction libre)}\fg{}.\\

L'Internet est l'utilisation la plus utile de tout le savoir de l'humanité, ne serait-ce que d'un point de vue biologique : pouvoir partager nos connaissances à une telle échelle permet d'améliorer globalement notre espèce en laissant à n'importe qui l'opportunité de pouvoir se les approprier. Le fait que tout le monde puisse publier ce qu'il veut a évidemment autant d'avantages que d'inconvénients, mais contrôler Internet est impensable et inutile, c'est à l'utilisateur de filtrer pour lui-même.\\

Selon moi, Internet pourrait servir à la démocratie de façon bien plus avancée que ce qui est déjà appliqué de nos jours : au lieu de pétitions, on pourrait utiliser des systèmes à vote de valeur sur des propositions politiques : la \og{}démocratie\fg{} représentative n'a plus de sens, elle est censée faciliter le dialogue, mais pourquoi le faciliter si tout le monde peut y prendre part sans problème majeur, la sécurisation peut être réalisée grâce à des technologies cryptographiques déjà utilisées de nos jours, même dans les cartes d'identité électroniques qui disposent d'un certificat permettant une signature numérique approuvée et vérifiée par le gouvernement lui-même.\\

Le problème, hélas, est que la société n'est pas encore prête à subir une telle révolution technologique. Les lois qui essaient de restreindre l'Internet, heureusement, ne font que le rendre plus fort, en améliorant les connaissances des utilisateurs quant aux moyens d'outrepasser la censure mais aussi de s'appliquer à une action civique, qui ne devrait d'ailleurs pas avoir à avoir lieu, en quoi des lois unanimement rejetées par le peuple et poussées par le seul lobbying peuvent avoir lieu dans une société qui se prétend démocratique ?\\

C'est en cela, je crois, que l'Internet est une révolution, il permet la transparence inhérente à un système informatique, et contamine notre société de cette caractéristique. Force est de le constater : notre société est loin d'être prête à fonctionner de façon démocratique (la démocratie étant un idéal impossible à atteindre, cela reste compréhensible), ceci, je l'espère, par manque d'implication du peuple, et non par absence complète de volonté de créer une société meilleure.\\

Internet reflète mes convictions du fait de sa totale invulnérabilité, ce que je vois en la liberté, et plus particulièrement la liberté d'expression et de rassemblement (\og{}virtuel\fg{}, certes, mais de rassemblement), en effet, la standardisation et la complète transparence du réseau lui permet de ne jamais sombrer, peu importe son état, il est impossible de l'éliminer complètement, même une guerre n'en a pas raison, puisqu'il suffit d'un seul lien, aussi lent soit-il, pour être connecté à tous les autres. Je suis convaincu de la pérennité des idées, et plus particulièrement de celles qu'on explicite sur le Net, car selon un vieil adage bien ancré dans les mentalités : \og{}the Internet never forgets\fg{}, et cela peut se vérifier par la lecture des archives quasiment complètes de Usenet maintenant universellement accessibles sur le Web, et au projet Wayback Machine de \url{archive.org}, qui recense une bonne partie du Web depuis sa création.\\

Pour le monde, Internet est maintenant devenu essentiel du fait de sa supériorité par rapport à tous les autres moyens de communication, mais peu de gens comprennent vraiment la philosophie intrinsèque au réseau et restent englués dans un schéma centralisé qui est contraire à sa structure, ce que Benjamin Bayart, le fondateur de French Data Network, un Fournisseur d'Accès à Internet français associatif, le plus ancien encore en activité en France, appelle non sans amertume le \emph{Minitel 2.0}, ce à quoi les gens accèdent via des terminaux stupides, s'en référant au contenu d'un seul serveur centralisant toutes les données, à la façon de tous les réseaux de communication précédents, selon un schéma obsolète et peu fiable : car, évidemment, si le centre n'est plus accessible, tout est perdu, les terminaux ne savent plus agir par eux-mêmes.\\

\part{Le pari}

Ce constat sur l'Internet m'amène au pari que je fais pour le siècle : j'espère que notre monde sera un jour géré en toute transparence et en démocratie participative par tous les citoyens, sans paternalisme étatique, sans corruption et même sans corruptibilité, à la façon de l'Internet, justement, où rien de vital n'est centralisé.\\

Ce défi est gigantesque, il implique que la population toute entière soit capable de gérer les infrastructures locales, de coordonner la gestion des infrastructures plus globales, et surtout, étant donné que le système que je tiens à voir apparaître est la démocratie participative : sans autorité autre que le consentement populaire, donc ni État, ni police, ni hiérarchie.\\

Les changements à apporter sont surtout d'ordre culturel, donc, il faut impliquer le citoyen lambda dans les affaires les plus diverses, l'intéresser à tout, et l'encourager à entreprendre et à s'attirer, ou non, un soutien de la part de ses pairs, le tout de manière raisonnée. Cette position est assez idéaliste, mais après tout réalisable sur le long terme : réformer l'État pour chaque fois lui retirer du pouvoir et le confier directement aux citoyens est réalisable, renverser l'État de façon violente serait une grosse perturbation inutile, peu importe que le système actuel me paraisse corrompu et complètement inefficace, je ne peux prôner la violence, elle n'amène que rarement des réformes raisonnables avec elle.\\

Mais ce pari comporte une composante qui m'inquiète quelque peu : il repose seulement sur l'initiative populaire, or, à travers l'Histoire, les peuples ont souvent fait preuve d'une formidable inertie et n'ont renversé un régime que pour en instaurer un autre très semblable en changeant les noms des fonctions et la classe sociale en charge. Et justement, le changement que je voudrais voir est le contraire, je voudrais l'inverse de notre système, où chaque individu a autant de pouvoir qu'un autre, et où la notion de régime se limite à la volonté du peuple. Le constat que l'on peut faire en Europe est qu'une couche de la jeunesse se sent trahie par une bureaucratie corrompue et mal huilée qui ne cesse de faire preuve d'une incompétence et d'une méconnaissance crasse sur de nombreux sujets, tandis que d'autres se plient sans rechigner à cet état de fait en constatant le confort qui leur est quand même accordé par le pouvoir en place.\\

Or, justement, que certains puissent plier et se soumettre au joug d'un pouvoir en qui ils ne peuvent avoir entièrement confiance me paraît absurde, il me paraît absurde qu'on puisse tolérer la représentation de ses idées politiques avec des programmes aussi pauvres que ceux des hommes et femmes politiques actuels, qui ne peuvent adresser tout ce que leur électorat, lui, adresse comme problèmes en réfléchissant à la politique, un seul cerveau ne peut en représenter des milliers, voire des millions, ni même deux.\\

Quand à mon rôle dans tout cela, je pense ne pas encore pouvoir me targuer de le connaître, mais je tiens à y réfléchir pendant les prochaines années, et essayer de sensibiliser la population à l'importance de la création d'un pouvoir populaire et à l'abandon de la bureaucratie, même si la tâche est totalement démentielle : comment convaincre une nation profondément ancrée dans le socialisme et l'institutionnalisation systématique de se défaire de toute hiérarchie forcée pour pouvoir libérer le peuple ?\\

Je suis donc plutôt pessimiste sur le court terme, la société actuelle ne verra pas beaucoup de changements de mon vivant, et j'en suis certain, y eût-il même une guerre ravageuse ou tout autre évènement fâcheux bouleversant complètement l'inconscient collectif, la nature humaine préfèrera la soumission à l'affirmation de l'individualité, je ne connais pas d'exemple pouvant prouver le contraire, et Internet ne peut changer les instincts après seulement quinze à vingt ans de diffusion à grande échelle.\\

Mais le futur n'est pas sans espoir, des partis politiques dont le programme est novateur (mais extrêmement incomplet et somme toute exécrable), comme le Parti Pirate allemand, commencent à gagner énormément d'influence proportionnellement à leur âge, et si la tendance se confirme, les idéaux politiques des partis dits \og{}traditionnels\fg{} seront-ils peut-être enfin considérés comme obsolètes et les idées des Lumières remises au goût du jour. Car après tout, comment peut-on se vanter de nos devises nationales quand elles disent \og{}l'Union fait la Force\fg{} ou \og{}Liberté, Égalité, Fraternité\fg{}, si on ne respecte en rien les idées qu'elles véhiculent ?\\

\part*{Licence}

Ce travail est libre, vous pouvez le modifier et/ou le redistribuer selon les termes de la Licence Art Libre, disponible sur le site de Copyleft Attitude \url{artlibre.org} et sur d'autres sites.

\end{document}

% Citations à inclure :
% - "The clothes make the man... Naked people have little or no influence in society" - Mark Twain
% -------------------------------------------- Finalement, non.
%
%Source de l'image :
%Patient, surrounded by men in white, receiving electric shock treatment.
%Location:	Worchester, MA, US
%Date taken:	August 1949
%Photographer:	Herbert Gehr
%http://i.imgur.com/tm9je.jpg 
