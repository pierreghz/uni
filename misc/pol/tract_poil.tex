%\documentclass[portrait,a0,draft]{a0poster} %change “draft” to “final” when done
%\documentclass[a4paper]{leaflet}
\documentclass[a4paper,11pt]{article}
\usepackage{xunicode}
\usepackage{fontspec}
\usepackage[frenchb]{babel}
%\usepackage{polyglossia}
%\usepackage[T1]{fontenc}
%\usepackage{graphicx}
%\usepackage{url}
\usepackage[absolute,overlay]{textpos}
%\oddsidemargin=-1cm
%\setlength{\textwidth}{6in}
%p\addtolength{\voffset}{-5pt}
%\linespread{1.3}
\widowpenalty=300
\clubpenalty=300
%\setmainlanguage{french}
%\setotherlanguage{english}


\begin{document}
\pagestyle {empty} %no headers, no footers, no numbers

\part*{Le poil féminin, ce bel inconnu.}\\
\\
Caractéristique sexuelle secondaire universellement dénigrée par les valeurs
occidentales, le poil féminin a pourtant des fonctions importantes : \\

\begin{itemize}
\item en augmentant la surface d’évaporation de la celle-ci, il permet
d’activement \emph{diminuer} la quantité de transpiration nécessaire au
refroidissement de la peau ;
\item du fait de la présence d’une glande sébacée à sa racine, il permet une
hydratation naturelle de la peau qui contribue à son élasticité et à sa
solidité ;
\item en tant que caractéristique sexuelle, il indique inconsciemment aux
individus du sexe opposé que vous êtes mature sexuellement, et plus une petite
fille innocente ;
\item il permet de capturer les phéromones pour faciliter et prolonger leur
diffusion, ce qui vous rend plus attirante sans effort conscient de la part de
quiconque ;
\item il est \emph{a fortiori} beau puisque présent naturellement sur le corps
féminin (pourriez-vous imaginer être repoussante par nature ?\footnote{Si oui,
vous devriez vous poser d’autres questions, plus graves, celles-là.}).
\end{itemize}
\\
Grâce à vos poils, vous pouvez affirmer votre puissance sexuelle sans aucun
investissement et sans aucune modification de votre corps, pourquoi
nécessairement s’en débarrasser ?

\begin{quote}
Les Occidentaux n'ont pas besoin de payer une police pour forcer les femmes à
obéir, il leur suffit de faire circuler les images pour que les femmes
s'esquintent à leur ressembler.
\end{quote}
Fatema \textsc{Mernissi} in \emph{Le Harem et l'Occident}

\end{document}

