\documentclass[10pt]{beamer}
\usepackage[utf8x]{inputenc}
\usepackage[T1]{fontenc}
\usepackage[francais]{babel}
\usepackage{graphicx}
\usepackage{amsmath}
\usepackage{amsfonts}
\usepackage{amssymb}
\usepackage{url}

\usetheme{default}
\usecolortheme{beaver}

\author{Pierre Ghazarian \and David Grommen}
\title{Présentation du tableau <<L'adoubement>>}
\institute{Collège Saint-Roch Ferrières}

\begin{document}

\frame{\titlepage}

\frame[plain]{
\begin{figure}[htb]

\begin{center}
\includegraphics[width=50mm]{Edmund_blair_leighton_accolade.jpg}
\end{center}

\caption{<<L'Adoubement>>, par Edmund Blair Leighton, 1901, collection privée}
\label{fig:tableau}
\end{figure}
}

\frame{
\frametitle{Qu'est-ce que l'adoubement ?}

Au Moyen Âge, l'adoubement était une cérémonie officielle à laquelle de 
nombreux nobles assistaient et qui consistait à consacrer un homme comme
chevalier du roi. Tout homme de bonne naissance, autrement dit riche et
descendant de suzerains, après avoir été page puis écuyer pouvait 
devenir chevalier. Pour ce faire, le père de l'enfant le confiait à 
une personne de confiance, un ami, ou un membre de sa famille qui 
devenait son parrain dès que l'enfant avait atteint l'âge de sept ans.
Il fallait que le père ait une confiance absolue en cette personne, 
le parrain, pour lui confier son enfant, car celui-ci devrait passer 
ses plus jeunes années sous sa garde et être élevé par lui.\\
\medskip
Source : Wikipédia (\url{http://fr.wikipedia.org/wiki/Adoubement}), 
consulté le 1\up{er} avril 2011.
\label{text:adoubement}

}

\frame{
\frametitle{Qui était Edmund Blair Leighton ?}
\begin{columns}[c] % the "c" option specifies center vertical alignment
\column{.5\textwidth} % column designated by a command
\includegraphics[scale=1]{portrait.jpg}
\column{.5\textwidth}
Edmund Blair Leighton (21 septembre 1852 — 1 septembre 1922) était un 
peintre anglais de scènes de genre historiques, spécialisé dans les sujets
de la Régence et médiévaux, associé librement au préraphaélisme.\\
\label{text:EBL}
\end{columns}

}

\frame{
\frametitle{Qu'est-ce que le préraphaélisme ?}
Le préraphaélisme est un mouvement artistique né au Royaume-Uni en 1848.
Ses membres constituaient une confrérie secrète, leur première exposition
eut lieu en 1849, ils rencontrèrent un succès mitigé durant les premières
années puis devinrent très populaires à la fin du XIX\ieme~siècle.

Les préraphaélites avaient, entre autres, pour dessein de rendre à l’art
un but fonctionnel et édifiant : leurs œuvres avaient pour fonction 
d’être morales. Mais cela n’excluait pas leur désir d’esthétisme. Le 
but de ces artistes était de s’adresser à toutes les facultés de 
l’Homme : son esprit, son intelligence, sa mémoire, sa conscience, 
son cœur… et non pas seulement à ce que l’œil voit.\\

Les préraphaélites aspiraient à agir sur les moeurs d’une société qui, à
leurs yeux, avait perdu tout sens moral depuis la Révolution 
industrielle. Cependant, « il ne suffit pas que l’art soit suggestif, 
soit didactique, soit moral, soit populaire ; il faut encore qu’il soit
national ».\\
\medskip
Source : Wikipédia (\url{http://fr.wikipedia.org/wiki/Préraphaélisme}),
consulté pour la dernière fois le 18 mai 2011.
\label{text:mouvement}
}
\end{document}
