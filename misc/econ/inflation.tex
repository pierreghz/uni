\documentclass[a4paper]{article}
\usepackage{fontspec}
%\usepackage[english]{babel}
\usepackage[margin=4cm]{geometry}
\usepackage{url}
\widowpenalty=300
\clubpenalty=300

\title{Inflation}
\author{\textsc{Ludwig von Mises}\\ \url{http://mises.org/daily/6294/Inflation}}
\date{Web: Monday, December 3, 2012\\
First published \emph{in} \emph{Economic Policy: Thoughts for Today and Tomorrow} (1979) from an original lecture held in 1958}

\begin{document}

\maketitle

If the supply of caviar were as plentiful as the supply of potatoes, the price 
of caviar — that is, the exchange ratio between caviar and money or caviar and 
other commodities — would change considerably. In that case, one could obtain 
caviar at a much smaller sacrifice than is required today. Likewise, if the 
quantity of money is increased, the purchasing power of the monetary unit 
decreases, and the quantity of goods that can be obtained for one unit of this 
money decreases also.\\

When, in the 16th century, American resources of gold and silver were 
discovered and exploited, enormous quantities of the precious metals were 
transported to Europe. The result of this increase in the quantity of money was
a general tendency toward an upward movement of prices in Europe. In the same 
way, today, when a government increases the quantity of paper money, the result
is that the purchasing power of the monetary unit begins to drop, and so prices
rise. This is called inflation.\\

Unfortunately, in the United States, as well as in other countries, some people
prefer to attribute the cause of inflation not to an increase in the quantity 
of money but, rather, to the rise in prices.\\

However, there has never been any serious argument against the economic 
interpretation of the relationship between prices and the quantity of money, or
the exchange ratio between money and other goods, commodities, and services. 
Under present-day technological conditions there is nothing easier than to 
manufacture pieces of paper upon which certain monetary amounts are printed. In
the United States, where all the notes are of the same size, it does not cost 
the government more to print a bill of a thousand dollars than it does to print
a bill of one dollar. It is purely a printing procedure that requires the same 
quantity of paper and ink.\\

In the 18th century, when the first attempts were made to issue bank notes and 
to give these bank notes the quality of legal tender — that is, the right to be
honored in exchange transactions in the same way that gold and silver pieces 
were honored — the governments and nations believed that bankers had some 
secret knowledge enabling them to produce wealth out of nothing. When the 
governments of the 18th century were in financial difficulties, they thought 
all they needed was a clever banker at the head of their financial management 
in order to get rid of all their difficulties.\\

Some years before the French Revolution, when the royalty of France was in 
financial trouble, the king of France sought out such a clever banker, and 
appointed him to a high position. This man was, in every regard, the opposite 
of the people who, up to that time, had ruled France. First of all he was not a
Frenchman, he was a foreigner — a Swiss from Geneva, Jacques Necker. Secondly, 
he was not a member of the aristocracy, he was a simple commoner. And, what 
counted even more in 18th-century France, he was not a Catholic but a 
Protestant. And so Monsieur Necker, the father of the famous Madame de Staël, 
became the minister of finance, and everyone expected him to solve the 
financial problems of France. But in spite of the high degree of confidence 
Monsieur Necker enjoyed, the royal cashbox remained empty — Necker's greatest 
mistake having been his attempt to finance aid to the American colonists in 
their war of independence against England without raising taxes. That was 
certainly the wrong way to go about solving France's financial troubles.\\

There can be no secret way to the solution of the financial problems of a 
government; if it needs money, it has to obtain the money by taxing its 
citizens (or, under special conditions, by borrowing it from people who have 
the money). But many governments, we can even say most governments, think there
is another method for getting the needed money; simply to print it.\\

If the government wants to do something beneficial — if, for example, it wants 
to build a hospital — the way to find the needed money for this project is to 
tax the citizens and build the hospital out of tax revenues. Then no special 
"price revolution" will occur, because when the government collects money for 
the construction of the hospital, the citizens — having paid the taxes — are 
forced to reduce their spending. The individual taxpayer is forced to restrict 
either his consumption, his investments, or his savings. The government, 
appearing on the market as a buyer, replaces the individual citizen: the 
citizen buys less, but the government buys more. The government, of course, 
does not always buy the same goods which the citizens would have bought; but on
the average there occurs no rise in prices due to the government's construction
of a hospital.\\

I choose this example of a hospital precisely because people sometimes say, "It
makes a difference whether the government uses its money for good or for bad 
purposes." I want to assume that the government always uses the money which it 
has printed for the best possible purposes — purposes with which we all agree. 
For it is not the way in which the money is spent, it is the way in which the 
government obtains this money that brings about those consequences we call 
inflation and which most people in the world today do not consider as 
beneficial.\\

For example, without inflating, the government could use the tax-collected 
money for hiring new employees or for raising the salaries of those who are 
already in government service. Then these people, whose salaries have been 
increased, are in a position to buy more. When the government taxes the 
citizens and uses this money to increase the salaries of government employees, 
the taxpayers have less to spend, but the government employees have more. 
Prices in general will not increase.\\

But if the government does not use tax money for this purpose, if it uses 
freshly printed money instead, it means that there will be people who now have 
more money while all other people still have as much as they had before. So 
those who received the newly printed money will be competing with those people 
who were buyers before. And since there are no more commodities than there were
previously, but there is more money on the market — and since there are now 
people who can buy more today than they could have bought yesterday — there 
will be an additional demand for that same quantity of goods. Therefore prices 
will tend to go up. This cannot be avoided, no matter what the use of this 
newly issued money will be.\\

And more importantly, this tendency for prices to go up will develop step by 
step; it is not a general upward movement of what has been called the "price 
level." The metaphorical expression "price level" must never be used.\\

When people talk of a "price level," they have in mind the image of a level of 
a liquid which goes up or down according to the increase or decrease in its 
quantity, but which, like a liquid in a tank, always rises evenly. But with 
prices, there is no such thing as a "level." Prices do not change to the same 
extent at the same time. There are always prices that are changing more 
rapidly, rising or falling more rapidly than other prices. There is a reason 
for this.\\

Consider the case of the government employee who received the new money added 
to the money supply. People do not buy today precisely the same commodities and
in the same quantities as they did yesterday. The additional money which the 
government has printed and introduced into the market is not used for the 
purchase of all commodities and services. It is used for the purchase of 
certain commodities, the prices of which will rise, while other commodities 
will still remain at the prices that prevailed before the new money was put on 
the market. Therefore, when inflation starts, different groups within the 
population are affected by this inflation in different ways. Those groups who 
get the new money first gain a temporary benefit.\\

When the government inflates in order to wage a war, it has to buy munitions, 
and the first to get the additional money are the munitions industries and the 
workers within these industries. These groups are now in a very favorable 
position. They have higher profits and higher wages; their business is moving. 
Why? Because they were the first to receive the additional money. And having 
now more money at their disposal, they are buying. And they are buying from 
other people who are manufacturing and selling the commodities that these 
munitions makers want.\\

These other people form a second group. And this second group considers 
inflation to be very good for business. Why not? Isn't it wonderful to sell 
more? For example, the owner of a restaurant in the neighborhood of a munitions
factory says, "It is really marvelous! The munitions workers have more money; 
there are many more of them now than before; they are all patronizing my 
restaurant; I am very happy about it." He does not see any reason to feel 
otherwise.\\

The situation is this: those people to whom the money comes first now have a 
higher income, and they can still buy many commodities and services at prices 
which correspond to the previous state of the market, to the condition that 
existed on the eve of inflation. Therefore, they are in a very favorable 
position. And thus inflation continues step by step, from one group of the 
population to another. And all those to whom the additional money comes at the 
early state of inflation are benefited because they are buying some things at 
prices still corresponding to the previous stage of the exchange ratio between 
money and commodities.\\

But there are other groups in the population to whom this additional money 
comes much, much later. These people are in an unfavorable position. Before the
additional money comes to them they are forced to pay higher prices than they 
paid before for some — or for practically all — of the commodities they wanted 
to purchase, while their income has remained the same, or has not increased 
proportionately with prices.\\

Consider for instance a country like the United States during the Second World 
War; on the one hand, inflation at that time favored the munitions workers, the
munitions industries, the manufacturers of guns, while on the other hand it 
worked against other groups of the population. And the ones who suffered the 
greatest disadvantages from inflation were the teachers and the ministers.\\

As you know, a minister is a very modest person who serves God and must not 
talk too much about money. Teachers, likewise, are dedicated persons who are 
supposed to think more about educating the young than about their salaries. 
Consequently, the teachers and ministers were among those who were most 
penalized by inflation, for the various schools and churches were the last to 
realize that they must raise salaries. When the church elders and the school 
corporations finally discovered that after all, one should also raise the 
salaries of those dedicated people, the earlier losses they had suffered still 
remained.\\

For a long time, they had to buy less than they did before, to cut down their 
consumption of better and more expensive foods, and to restrict their purchase 
of clothing — because prices had already adjusted upward, while their incomes, 
their salaries, had not yet been raised. (This situation has changed 
considerably today, at least for teachers.)\\

There are therefore always different groups in the population being affected 
differently by inflation. For some of them, inflation is not so bad; they even 
ask for a continuation of it because they are the first to profit from it. We 
will see, in the next lecture, how this unevenness in the consequences of 
inflation vitally affects the politics that lead toward inflation.\\

Under these changes brought about by inflation, we have groups who are favored 
and groups who are directly profiteering. I do not use the term "profiteering" 
as a reproach to these people, for if there is someone to blame, it is the 
government that established the inflation. And there are always people who 
favor inflation, because they realize what is going on sooner than other people
do. Their special profits are due to the fact that there will necessarily be 
unevenness in the process of inflation.\\

The government may think that inflation — as a method of raising funds — is 
better than taxation, which is always unpopular and difficult. In many rich and
great nations, legislators have often discussed, for months and months, the 
various forms of new taxes that were necessary because the parliament had 
decided to increase expenditures. Having discussed various methods of getting 
the money by taxation, they finally decided that perhaps it was better to do it
by inflation.\\

But of course, the word "inflation" was not used. The politician in power who 
proceeds toward inflation does not announce, I am proceeding toward inflation."
The technical methods employed to achieve the inflation are so complicated that
the average citizen does not realize inflation has begun.\\

One of the biggest inflations in history was in the German Reich after the 
First World War. The inflation was not so momentous during the war; it was the 
inflation after the war that brought about the catastrophe. The government did 
not say, "We are proceeding toward inflation." The government simply borrowed 
money very indirectly from the central bank. The government did not have to ask
how the central bank would find and deliver the money. The central bank simply 
printed it.\\

Today the techniques for inflation are complicated by the fact that there is 
checkbook money. It involves another technique, but the result is the same. 
With the stroke of a pen, the government creates fiat money, thus increasing 
the quantity of money and credit. The government simply issues the order, and 
the fiat money is there.\\

The government does not care, at first, that some people will be losers, it 
does not care that prices will go up. The legislators say, "This is a wonderful
system!" But this wonderful system has one fundamental weakness: it cannot 
last. If inflation could go on forever, there would be no point in telling 
governments they should not inflate. But the certain fact about inflation is 
that, sooner or later, it must come to an end. It is a policy that cannot last.\\

In the long run, inflation comes to an end with the breakdown of the currency; 
it comes to a catastrophe, to a situation like the one in Germany in 1923. On 
August 1, 1914, the value of the dollar was four marks and twenty pfennigs. 
Nine years and three months later, in November 1923, the dollar was pegged at 
4.2 trillion marks. In other words, the mark was worth nothing. It no longer 
had any value.\\

Some years ago, a famous author, John Maynard Keynes, wrote, "In the long run 
we are all dead." This is certainly true, I am sorry to say. But the question 
is, how short or long will the short run be? In the 18th century there was a 
famous lady, Madame de Pompadour, who is credited with the dictum "Après nous 
le déluge" ("After us will come the flood"). Madame de Pompadour was happy 
enough to die in the short run. But her successor in office, Madame du Barry, 
outlived the short run and was beheaded in the long run. For many people the 
"long run" quickly becomes the "short run" — and the longer inflation goes on 
the sooner the "short run."\\

How long can the short run last? How long can a central bank continue an 
inflation? Probably as long as people are convinced that the government, sooner
or later, but certainly not too late, will stop printing money and thereby stop
decreasing the value of each unit of money.\\

When people no longer believe this, when they realize that the government will 
go on and on without any intention of stopping, then they begin to understand 
that prices tomorrow will be higher than they are today. Then they begin buying
at any price, causing prices to go up to such heights that the monetary system 
breaks down.\\

I refer to the case of Germany, which the whole world was watching. Many books 
have described the events of that time. (Although I am not a German, but an 
Austrian, I saw everything from the inside: in Austria, conditions were not 
very different from those in Germany; nor were they much different in many 
other European countries.) For several years, the German people believed that 
their inflation was just a temporary affair, that it would soon come to an end.
They believed it for almost nine years, until the summer of 1923. Then, 
finally, they began to doubt. As the inflation continued, people thought it 
wiser to buy anything available, instead of keeping money in their pockets. 
Furthermore, they reasoned that one should not give loans of money, but on the 
contrary, that it was a very good idea to be a debtor. Thus inflation continued
feeding on itself.\\

And it went on in Germany until exactly November 20, 1923. The masses had 
believed inflation money to be real money, but then they found out that 
conditions had changed. At the end of the German inflation, in the fall of 
1923, the German factories paid their workers every morning in advance for the 
day. And the workingman who came to the factory with his wife, handed his wages
— all the millions he got — over to her immediately. And the lady immediately 
went to a shop to buy something, no matter what. She realized what most people 
knew at that time — that overnight, from one day to another, the mark lost 50\% 
of its purchasing power. Money, like chocolate in a hot oven, was melting in 
the pockets of the people. This last phase of German inflation did not last 
long; after a few days, the whole nightmare was over: the mark was valueless 
and a new currency had to be established.\\

Lord Keynes, the same man who said that in the long run we are all dead, was 
one of a long line of inflationist authors of the 20th century. They all wrote 
against the gold standard. When Keynes attacked the gold standard, he called it
a "barbarous relic." And most people today consider it ridiculous to speak of a
return to the gold standard. In the United States, for instance, you are 
considered to be more or less a dreamer if you say, "Sooner or later, the 
United States will have to return to the gold standard."\\

Yet the gold standard has one tremendous virtue: the quantity of money under 
the gold standard is independent of the policies of governments and political 
parties. This is its advantage. It is a form of protection against spendthrift 
governments. If, under the gold standard, a government is asked to spend money 
for something new, the minister of finance can say, "And where do I get the 
money? Tell me, first, how I will find the money for this additional 
expenditure."\\

Under an inflationary system, nothing is simpler for the politicians to do than
to order the government printing office to provide as much money as they need 
for their projects. Under a gold standard, sound government has a much better 
chance; its leaders can say to the people and to the politicians, "We can't do 
it unless we increase taxes."\\

But under inflationary conditions, people acquire the habit of looking upon the
government as an institution with limitless means at its disposal: the state, 
the government, can do anything. If, for instance, the nation wants a new 
highway system, the government is expected to build it. But where will the 
government get the money?\\

One could say that in the United States today — and even in the past, under 
McKinley — the Republican Party was more or less in favor of sound money and of
the gold standard, and the Democratic Party was in favor of inflation, of 
course not a paper inflation, but a silver inflation.\\

It was, however, a Democratic president of the United States, President 
Cleveland, who at the end of the 1880s vetoed a decision of Congress, to give a
small sum — about \$10,000 — to help a community that had suffered some 
disaster. And President Cleveland justified his veto by writing: "While it is 
the duty of the citizens to support the government, it is not the duty of the 
government to support the citizens." This is something which every statesman 
should write on the wall of his office to show to people who come asking for 
money.\\

I am rather embarrassed by the necessity to simplify these problems. There are 
so many complex problems in the monetary system, and I would not have written 
volumes about them if they were as simple as I am describing them here. But the
fundamentals are precisely these: if you increase the quantity of money, you 
bring about the lowering of the purchasing power of the monetary unit. This is 
what people whose private affairs are unfavorably affected do not like. People 
who do not benefit from inflation are the ones who complain.\\

If inflation is bad and if people realize it, why has it become almost a way of
life in all countries? Even some of the richest countries suffer from this 
disease. The United States today is certainly the richest country in the world,
with the highest standard of living. But when you travel in the United States, 
you will discover that there is constant talk about inflation and about the 
necessity to stop it. But they only talk; they do not act.\\

To give you some facts: after the First World War, Great Britain returned to 
the prewar gold parity of the pound. That is, it revalued the pound upward. 
This increased the purchasing power of every worker's wages. In an unhampered 
market the nominal money wage would have fallen to compensate for this and the 
workers' real wage would not have suffered. We do not have time here to discuss
the reasons for this. But the unions in Great Britain were unwilling to accept 
an adjustment of money wage rates downward as the purchasing power of the 
monetary unit rose. Therefore real wages were raised considerably by this 
monetary measure. This was a serious catastrophe for England, because Great 
Britain is a predominantly industrial country that has to import its raw 
materials, half-finished goods, and food stuffs in order to live, and has to 
export manufactured goods to pay for these imports. With the rise in the 
international value of the pound, the price of British goods rose on foreign 
markets and sales and exports declined. Great Britain had, in effect, priced 
itself out of the world market.\\

The unions could not be defeated. You know the power of a union today. It has 
the right, practically the privilege, to resort to violence. And a union order 
is, therefore, let us say, not less important than a government decree. The 
government decree is an order for the enforcement of which the enforcement 
apparatus of the government — the police — is ready. You must obey the 
government decree, otherwise you will have difficulties with the police.\\

Unfortunately, we have now, in almost all countries all over the world, a 
second power that is in a position to exercise force: the labor unions. The 
labor unions determine wages and then strike to enforce them in the same way in
which the government might decree a minimum wage rate. I will not discuss the 
union question now; I shall deal with it later. I only want to establish that 
it is the union policy to raise wage rates above the level they would have on 
an unhampered market. As a result a considerable part of the potential labor 
force can be employed only by people or industries that are prepared to suffer 
losses. And, since businesses are not able to keep on suffering losses, they 
close their doors and people become unemployed. The setting of wage rates above
the level they would have on the unhampered market always results in the 
unemployment of a considerable part of the potential labor force.\\

In Great Britain, the result of high wage rates enforced by the labor unions 
was lasting unemployment, prolonged year after year. Millions of workers were 
unemployed, production figures dropped. Even experts were perplexed. In this 
situation the British government made a move which it considered an 
indispensable, emergency measure: it devalued its currency.\\

The result was that the purchasing power of the money wages, upon which the 
unions had insisted, was no longer the same. The real wages, the commodity 
wages, were reduced. Now the worker could not buy as much as he had been able 
to buy before, even though the nominal wage rates remained the same. In this 
way, it was thought, real wage rates would return to free market levels and 
unemployment would disappear.\\

This measure — devaluation — was adopted by various other countries, by France,
the Netherlands, and Belgium. One country even resorted twice to this measure 
within a period of one year and a half. That country was Czechoslovakia. It was
a surreptitious method, let us say, to thwart the power of the unions. You 
could not call it a real success, however.\\

After a few years, the people, the workers, even the unions, began to 
understand what was going on. They came to realize that currency devaluation 
had reduced their real wages. The unions had the power to oppose this. In many 
countries they inserted a clause into wage contracts providing that money wages
must go up automatically with an increase in prices. This is called indexing. 
The unions became index conscious. So, this method of reducing unemployment 
that the government of Great Britain started in 1931 — which was later adopted 
by almost all important governments — this method of "solving unemployment" no 
longer works today.\\

In 1936, in his \emph{General Theory of Employment, Interest and Money}, Lord Keynes 
unfortunately elevated this method — the emergency measures of the period 
between 1929 and 1933 — to a principle, to a fundamental system of policy. And 
he justified it by saying, in effect, "Unemployment is bad. If you want 
unemployment to disappear you must inflate the currency."\\

He realized very well that wage rates can be too high for the market, that is, 
too high to make it profitable for an employer to increase his work force, thus
too high from the point of view of the total working population, for with wage 
rates imposed by unions above the market only a part of those anxious to earn 
wages can obtain jobs.\\

And Keynes said, in effect, "Certainly mass unemployment prolonged year after 
year, is a very unsatisfactory condition." But instead of suggesting that wage 
rates could and should be adjusted to market conditions, he said, in effect, 
"If one devalues the currency and the workers are not clever enough to realize 
it, they will not offer resistance against a drop in real wage rates, as long 
as nominal wage rates remain the same." In other words, Lord Keynes was saying 
that if a man gets the same amount of sterling today as he got before the 
currency was devalued, he will not realize that he is, in fact, now getting 
less.\\

In old fashioned language, Keynes proposed cheating the workers. Instead of 
declaring openly that wage rates must be adjusted to the conditions of the 
market — because, if they are not, a part of the labor force will inevitably 
remain unemployed — he said, in effect, "Full employment can be reached only if
you have inflation. Cheat the workers." The most interesting fact, however, is 
that when his \emph{General Theory} was published, it was no longer possible to cheat,
because people had already become index conscious. But the goal of full 
employment remained.\\

What does "full employment" mean? It has to do with the unhampered labor 
market, which is not manipulated by the unions or by the government. On this 
market, wage rates for every type of labor tend to reach a point at which 
everybody who wants a job can get one and every employer can hire as many 
workers as he needs. If there is an increase in the demand for labor, the wage 
rate will tend to be greater, and if fewer workers are needed, the wage rate 
will tend to fall.\\

The only method by which a "full employment" situation can be brought about is 
by the maintenance of an unhampered labor market. This is valid for every kind 
of labor and for every kind of commodity.\\

What does a businessman do who wants to sell a commodity for five dollars a 
unit? When he cannot sell it at that price, the technical business expression 
in the United States is, "the inventory does not move." But it must move. He 
cannot retain things because he must buy something new; fashions are changing. 
So he sells at a lower price. If he cannot sell the merchandise at five 
dollars, he must sell it at four. If he cannot sell it at four, he must sell it
at three. There is no other choice as long as he stays in business. He may 
suffer losses, but these losses are due to the fact that his anticipation of 
the market for his product was wrong.\\

It is the same with the thousands and thousands of young people who come every 
day from the agricultural districts into the city trying to earn money. It 
happens so in every industrial nation. In the United States they come to town 
with the idea that they should get, say, \$100 a week. This may be impossible. 
So if a man cannot get a job for \$100 a week, he must try to get a job for \$90 
or \$80, and perhaps even less. But if he were to say — as the unions do — "\$100
a week or nothing," then he might have to remain unemployed. (Many do not mind 
being unemployed, because the government pays unemployment benefits — out of 
special taxes levied on the employers — which are sometimes nearly as high as 
the wages the man would receive if he were employed.)\\

Because a certain group of people believes that full employment can be attained
only by inflation, inflation is accepted in the United States. But people are 
discussing the question, Should we have a sound currency with unemployment, or 
inflation with full employment? This is in fact a very vicious analysis.\\

To deal with this problem we must raise the question, How can one improve the 
condition of the workers and of all other groups of the population? The answer 
is, by maintaining an unhampered labor market and thus achieving full 
employment. Our dilemma is, shall the market determine wage rates or shall they
be determined by union pressure and compulsion? The dilemma is not "shall we 
have inflation or unemployment?"\\

This mistaken analysis of the problem is argued in England, in European 
industrial countries and even in the United States. And some people say, "Now 
look, even the United States is inflating. Why should we not do it also."\\

To these people one should answer first of all, "One of the privileges of a 
rich man is that he can afford to be foolish much longer than a poor man." And 
this is the situation of the United States. The financial policy of the United 
States is very bad and is getting worse. Perhaps the United States can afford 
to be foolish a bit longer than some other countries.\\

The most important thing to remember is that inflation is not an act of God; 
inflation is not a catastrophe of the elements or a disease that comes like the
plague. Inflation is a policy — a deliberate policy of people who resort to 
inflation because they consider it to be a lesser evil than unemployment. But 
the fact is that, in the not very long run, inflation does not cure 
unemployment.\\

Inflation is a policy. And a policy can be changed. Therefore, there is no 
reason to give in to inflation. If one regards inflation as an evil, then one 
has to stop inflating. One has to balance the budget of the government. Of 
course, public opinion must support this; the intellectuals must help the 
people to understand. Given the support of public opinion, it is certainly 
possible for the people's elected representatives to abandon the policy of 
inflation.\\

We must remember that, in the long run, we may all be dead and certainly will 
be dead. But we should arrange our earthly affairs, for the short run in which 
we have to live, in the best possible way. And one of the measures necessary 
for this purpose is to abandon inflationary policies.\\


\emph{Ludwig von Mises was the acknowledged leader of the Austrian School of economic
thought, a prodigious originator in economic theory, and a prolific author. 
Mises's writings and lectures encompassed economic theory, history, 
epistemology, government, and political philosophy. His contributions to 
economic theory include important clarifications on the quantity theory of 
money, the theory of the trade cycle, the integration of monetary theory with 
economic theory in general, and a demonstration that socialism must fail 
because it cannot solve the problem of economic calculation. Mises was the 
first scholar to recognize that economics is part of a larger science in human 
action, a science that Mises called "praxeology." See Ludwig von Mises's 
article archives \footnote{\url{http://mises.org/daily/author/280/Ludwig-von-Mises}}.}\\

Copyright © 2012 by the Ludwig von Mises Institute. Permission to reprint in 
whole or in part is hereby granted, provided full credit is given.

\end{document}
